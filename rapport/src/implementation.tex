\subsection{Formatage des données}
\par Afin d'améliorer les performances des différents blocs de cette implémentation, nous avons choisi de formater au mieux les données. Uniquement les tweets sont concernés pas ce formatage.

\par Plusieurs méthodes ont été appliquées:
\begin{itemize}
  \item transformation en minuscule de tous les caractères en majuscule.
  \item suppression du hashag \#semst présent à la fin de chaque tweet, et n'apportant aucun caractère informatif au tweets.
  \item remplacement des symboles pouvant être exprimés par des mots en ledit mot. Par exemple, au symbole \% correspond le mot "percent", au symbole \& le mot "and", ...
  \item suppression de tous les symboles ne pouvant pas être exprimés pas des mots, en particulier les symboles de ponctuation.
  \item transformation des formes contractées en leur forme étendue. Par exemple, à "ma'am" correspond le mot "madam", à "I'm" les mots "I am".
  \item transformation des nombres en le mot "number". En effet, la valeur exprimée n'a pas d'influence sur le sens du tweet.
  \item transformation des mentions (utilisation du symbole @) d'utilisateurs, en le mot "someone". Dans certains contextes, la valeur de cette mention pourrait être intéressante (dans le cas ou on arriverait à déterminer si cette mention a une valeur positive ou non par exemple), mais de part l'implémentation que nous avons choisi, ce formatage est plus pertinent, et améliore l'apprentissage du bloc "Opinion towards" detection.
  \item transformation de tous les verbes en leur forme infinitive.
\end{itemize}

\par Quelques autres formatages des tweets pourraient être pertinents dans le cadre de ce projet:
\begin{itemize}
  \item si plusieurs mots ont un synonyme en commun, il pourrait s'avérer judicieux de remplacer lesdits mots par ce synonyme. Ainsi, nous réduisons le dictionnaire de mots utilisés, et pouvons améliorer l'apprentissage du bloc "Opinion towards" detection.
  \item il est compliqué d'extraire le sens des hashags utilisés dans les tweets, cependant il correspondent régulièrement à un groupe de mots collés. Ainsi un hashag comme \#NetNeutrality pourrait être décomposé en deux mots, "net" et "neutrality".
  \item transformation des mots au pluriel en leur forme au singulier. En effet, ils seront interprétés comme étant des mots différents par le bloc "Opinion towards" detection, alors qu'ils n'influent pas sur sens général du tweet.
  \item suppression des répétitions de mots.
\end{itemize}
Certains formatage appliqués ne sont pas forcément des plus précis, et pourraient être améliorés. Par exemple, dans le cas du remplacement des contractions en mots, il est difficile de déterminer si "he's" correspond à "he is" ou "he has".

\subsection{Sentiment analysis}

\subsection{"Opinion towards" detection}

\subsection{Stance detection}
