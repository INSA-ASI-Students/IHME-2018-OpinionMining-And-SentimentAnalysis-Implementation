	\par Ce document concerne la présentation de réalisation du projet dans le cadre de l'EC Interaction Homme-Machines Évoluées.  
	\par L'objectif de ce projet est de réaliser une analyse de tweet afin d'en extraire le stance, c'est à dire déterminer si l'auteur du tweet était 'POUR', 'CONTRE' ou 'NEUTRE' via à vis d'un sujet donnée.\\ 
	
	\par Pour réaliser notre projet nous nous sommes basés sur les données du concours international SemEval afin de comparer nos résultats à ceux des autres participants.
	\par Nous avions à notre disposition deux jeux de données concenant les informations suivantes : le tweet, le sujet du tweet, l'opinion toward (c'est à dire si le tweet à un lien direct ou non avec le sujet), le sentiment (positif, négatif ou neutre) et le stance du tweet (pour, contre, neutre). 
	
	\par Notre choix d'implémentation pour notre sujet a été de réaliser 3 briques logicielles permettant chacune de déterminer, le sentiment du tweet, l'opinion toward et le stance. \\
	
	\par Dans un premier temps nous vous présenterons la conception réalisé sur ce sujet et ensuite nous vous détaillerons les choix et le résultat des implémentations réalisés pour nos 3 briques logicielles.
